Starcraft broodwar is an old real time strategy game, but it is still a intersant game, because there are big turnements. Another intersant thing about starcraft
broodwar is it three races are balaced. The three races are the insectoid Zerg, the advanced alien race Protoss and the Human-like Terran. The three races are all unique
and excel at different skills, which requires players to adapt their strategies depending on which race they are facing. Starcraft broodwar \\

This makes it ideal to make a bot for starcraft broodwar. The game is a \abrts game, which requires a player to build an army which can defeat the opponent.
So controlling the units in a battle is a big part of starcraft. This task eee a lot of effert from a human, which only can do a limet number of actions per minute. A bot
is very different on this issus, because it can perform a high amount of actions per minute. The bot can also make calculations while it is in battle with high speed and precession.
Meanwhile stil keeping it's buildings working. This is why a the report will focus on controlling one unit type. Even though controlling a unit type is very ideal and is the focus of
the project, there are another very imported part of a starcraft game, because starcraft is also about adapting to the surrondings and the opponents build order (which we will
explean in depth latter in the report). That is why our bot is going to be eable to do and detact build orders. It is our belives that our bot will be eable to control a  smaller army
compared with a human's army, and still win, because it will be eable to learn to attack the optimal way.


\section{Purpose}
	The purpose of this project is to create a computer-bot which can play Starcraft Broodwar. 
	The bot is going to learn how to control a unit type, so that it will kill the opponent's army it else wouldn't have killed.
	The bot should be enable to react to what the opponent does by applying scouting.

\section{Problem Statement}
	Our goals for this project is to:
	\begin{itemize}
		\item Create an intelligent bot for the game Starcraft Broodwar
		\item Apply machine intelligence theory in the modeling of the bot
		\item Make a bot that can learn by playing Starcraft
		\item Make sure the bot analysis the battles and know if it should pull back.
		\item The bot should know a sample of built orders, so it can use the best strategy
		\item Make a bot that can compare the known variables from scouting and chose the best strategy
	\end{itemize}

\section{Overview}
	A short overview of the report where one can read what the different chapters contain

\subsection{Chapter 1 - }
Introduction to the project explaining how human makes decision and how a computer would make decisions. This part also explains the purpose of this project, and what kind of problem we would like to solve.
\subsection{Chapter 2}
Here we explain the basic rules of the game, what is important to obtain to win a match and how one can win over an opponent. We mention some different areas of the game one has to master to be good or even perfect to win, we also mention some different standard attack methods one can use to slow down the opponent. At last we analysis what kind of unit we want a computer to control.
\subsection{Chapter 3}
The design part is about different design methods one can use to control several units on a map, to avoid obstacles and still be drawn to the opponent. The last part is about the theory of agent learning what different kinds we have looked upon.
\subsection{Chapter 4}
This chapter is where the theory is being used in practice, and here we explain how we have implemented the potential fields, and how we have implemented the learning part of the agent.
\subsection{Chapter 5}
This chapter have the purpose to test our bot. We do this by having three test where each test is run on two different maps. The first test is the case
where the enemy attack the vultures without the vultures being controlled. This gives an idea of how strong the two forces are comparet with each other. The next test is the improvements our bot will have by using potential fields. The last test is about our bots ability to learn by using reinforcement learning.
\subsection{Chapter 6}
This chapter is where we will conclude the impact the machine intelligence have had on our bot and the results we have gathered from our bot.
\subsection{Chapter 7}