In this project we look at Starcraft Broodwar which is a \abrts game and Machine Intelligence. Starcraft Broodwar is interesting in relation to look at for several reasons, it requires as much strategy as chess, it has the unknown aspect of poker, and its as unpredictable as scrabble. These three factors makes it an ideal game for Machine Intelligence.

Another advantages of Starcraft Broodwar is that its easy to learn but hard to master. The basic of the game is very simple can be understood in a matter 30 minutes, but to become a true professional take years of intensive training. Which is great for Machine Intelligence since we can easily teach a computer to play the game, but making it good will be hard work.\\

Using a computer to play Starcraft Broodwar is interesting in the sense that computer exceeds humans in more then one area. It will be able to do many advanced calculation during a game, and be able to control the mouse and keyboard very rapidly manner. Especially the mouse and keyboard movement opens the door for some interesting opportunities, a human can only control a limited about of units or buildings a one time, but a computer would be able to control all of them. \\

Machine Intelligence is interesting in relation to Starcraft Broodwar because we would be able to take a subset of the game, and really focus on that part, and train the computer to do that one part just right each time. Which could be
\begin{itemize}
	\item Teach it to use just the right strategy at the right time.
	\item Predict what the other players are during.
	\item Teach it to control single units in just the right way.
\end{itemize}

All the above part truly makes Starcraft Broodwar an interesting game to look at with a lot of different options for the main focus of the remainder of this report.

\section{Purpose}

	The purpose of this project is to create a computer program that is able to play a full game of Starcraft Broodwar. But also showing the advantages of using a computer by making it due tasks that would be impossible for a human being. It should be able, trough trials, learn from its mistakes and become better using machine intelligence theory. 


\section{Problem Statement}
	Our goals for this project is to:
	\begin{itemize}
		\item Create an intelligent bot for the game Starcraft Broodwar
		\item Apply Machine Intelligence theory in the modeling of the bot
		\item Make a bot that can improve by playing Starcraft
	\end{itemize}

\section{Overview}
	We will now give a brief overview of the different chapters of the report.

\paragraph*{Chapter 2 - Analysis}
The analysis begins with an introduction to the basic rules for playing a game of Starcraft Broodwar. Then we go on to explain some of the most common Terran tactics, where we talk about the importance of information and the balance between macro and micro. Then we will take a look at what at how Machine Intelligence can be used to make a bot and what aspect of the game which can benefit the most from Machine Intelligence. The last thing in the analysis will be a analysis of the different units available for Terran and there strengths.

\paragraph*{Chapter 3 - Design}
In the design chapter we begin by describing the different criteria we have for our bot. Then we go on to explain how we designed the different managers we use for various tasks like building and scouting. Then we give an overview of what potential fields are, and how we are going to use them for movement. Afterwards we describe different kinds of agent learning and how we can use it to improve potential fields. At the end we talk about Bayesian networks and decision trees, and how we can use Bayesian networks to do predictions.

\paragraph*{Chapter 4 - Implementation}
In this chapter we begin by looking at how we implemented the Bayesian networks, and how we insert evidence on the networks. Then we talk about the implemented of the managers and explain the most important parts of the code. Then we look at potential fields how the differ from the design and they are implemented. At the end we look at how Q-Learning is implemented.

\paragraph*{Chapter 5 - Tests}
In chapter 5 we start by during a few basic test to compare against the potential fields. First how does the build in bot due against it self. Then how does the our potential field without any learning do against the build in bot. Then we look at how different alpha and gamma values effects the way our bot learn and how this effects its performance. After this we test the different Bayesian networks and how they do at prediction the enemies spawn, prediction the enemies build order and the enemies threat level.

\paragraph*{Chapter 6 - Conclusion}
In this chapter we conclude on this project.

\paragraph*{Chapter 7 - Future Work}
This chapter focus on different action we could take to improve our work if we had more time.
