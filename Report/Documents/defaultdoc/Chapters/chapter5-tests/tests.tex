In this Chapter we will describe the test results of mirco battles to test our AI. We have two test maps, which all our test are going to be performed in.
The first map is  zerglings against vultures and the other map is marines against vultures. We have the first map, because zerglings have a lower movement
speed then vultures, so the vultures should have be eable to win the battle with out losses. In the second map is more interesting, because the marines
are ranged units, this means we can see how well our bot do comperted with the built-in bot. The maps are  designed so that the player with the vultures
are weaker then the other player. This means our bot can improve and should be better then the built-in bot. In the first map it is 20 marines against 5
vultures and the other map it is 30 zerglinges against 5 vultures. Each test will be preformed on both maps. \\

 In the first test we will make the build in AI fight it self, to know how the vultures are doing, and have a base case to compare with. This will help us understand how strong the
two armys are compert with each other and what losses we can expect. The other test will test the potential fields and what effact they have on the result.
In order to test this we will look at the difference with potential fields and without them. The last test we will test reinforcement learning and see how
fast the bot will improve over time. We will do this by comparing serveral test runs.

\section{First Test}
So in this test the two forces are just attacking each other. This means the vultures are losing in both maps. Below is a list of the resultes from the test
in the two maps.

\begin{centering}
 \begin{tabular}{|l||c|c|c|c|}
	\multicolumn{5}{c}{First Test Resultes From First Map} \\
	\hline
	\ Players & Produced units & Killed Unit & Lost Units\\
	\hline
	\hline
		Player with vultures & 5 & 9 & 5 \\
	\hline
		Player with zerglings & 30 & 9 & 5\\
	\hline

\end{tabular}
\end{centering}

\begin{centering}
 \begin{tabular}{|l||c|c|c|c|}
	\multicolumn{5}{c}{First Test Resultes From Second Map} \\
	\hline
	\ Players & Produced units & Killed Unit & Lost Units \\
	\hline
	\hline
		Player with vultures & 5 & 5 & 5\\
	\hline
		Player with marines & 20 & 5 & 5\\
	\hline

\end{tabular}
\end{centering}

After this first test we know how weaker the player with the vultures are compared with the opponent.

\section{Second Test}
This is the first test where we test the our bot agenst the built-in bot. 

\begin{centering}
 \begin{tabular}{|l||c|c|c|c|}
	\multicolumn{5}{c}{First Test Resultes From Second Map} \\
	\hline
	\ Players & Produced units & Killed Unit & Lost Units\\
	\hline
	\hline
		Player with vultures & 5 & 30 & 0\\
	\hline
		Player with zerglings & 30 & 0 & 30\\
	\hline

\end{tabular}
\end{centering}

Out bot get this impressive result by not overextending himself and thereby only be in sight of a few zerglinges. This forced some zerglinges out
of the group and made them a easy target. Though out the match this patten repeated, until the point where there was no more zerglings and all
five vulture was at full health.

\begin{centering}
 \begin{tabular}{|l||c|c|c|c|}
	\multicolumn{5}{c}{First Test Resultes From Second Map} \\
	\hline
	\ Players & Produced units & Killed Unit & Lost Units\\
	\hline
	\hline
		Player with vultures & 5 & 6 & 5\\
	\hline
		Player with marines & 20 & 5 & 6\\
	\hline

\end{tabular}
\end{centering}


The result from this test wasn't as good as you would expect. The result of the vultures moving back and forth should be way better then static movment.
But the fact is that the vultures only killed one more marine. The reason why this is the case is that the vultures couldn't use the same attack patteren as
effektive as with the zerglinges, because the marines range attack sometimes delt damage to the vultures. So the vultures is getting to close before attacking.
We see from this test that potentil feild in it self is not enough, if we like to win or have less losse then the opponent. In the next test we will test if there is an
improvment if the bot uses reinforcement learning.
