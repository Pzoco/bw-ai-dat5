\section{Learning rate}
This section is about tests with the learning rate of the agent. The more thorough explanation of the generalization can be read here \ref{qlearning}.
By changing the $\gamma$ and $\alpha$ values the agent will learn differently - so we have made a test where we have run between 15000 to 100000 iterations. All the test have been run with 5 vultures against 12 marines. The graphs from the test can be seen in full size in the appendix \ref{appendix}. We will make test over how much damage the vultures have dealt and how many units we have lost and how many units we have killed. As mentioned before will these test be performed with different learning values. It is worth mentioning that 5 vultures against 12 marines is a very hard battle for the vultures and the marines are favoured to win that battle.



%%%% TEST WITH A9 G2 %%%%
\subsection*{Learning rate test 1.1}
In this test we are using the values $\alpha$ 9 and $\gamma$ 2 and the agent has run 40182 iterations


\textit{The figures showing in the test, can be watched full sized in the appendix \ref{appendix}} 
%Damage given and taken
\insertmarginfigure{height=3in}{learningrate/A9G2/damage.png}
			{Alpha 9 Gamma 2 Damage - Yellow: Damage given - Red: Damage taken}{fig:a9g2_damage}{-3in}

In figure \ref{fig:a9g2_damage} one can see every time the y-axis peaks (yellow graph) the agent have killed all of the opponent as we have hoped for. The red graph is damage taken, if it's on 400 all the 5 vultures have died and the agent have had a lost.

%%%%%AVERAGE%%%%%
\begin{centering}
 \begin{tabular}{|l||c|c|c|}
	\multicolumn{4}{c}{Average results from test 1.1} \\
	\hline
		Damage taken & Damage given & Units lost & Enemies killed\\
	\hline
		392,71 & 337,26 & 4,79 & 7,51 \\
	\hline
\label{test1.1}
\end{tabular}
\end{centering}

%%%AVERAGE%%%





%Units killed and units lost
\insertmarginfigure{height=3in}{learningrate/A9G2/units_lost_and_killed.png}
			{Alpha 9 Gamma 2 Lost and killed - Yellow: Enemies killed - Red: Vultures left}{fig:a9g2_lak}{-3in}
In figure \ref{fig:a9g2_lak} the yellow graph is how many marines was killed, and when it peaks to the 12 mark our agent have won a battle






%%%% TEST WITH A6 G4 %%%%
\subsection*{Learning rate test 1.2}
In this test we are using the values $\alpha$ 6 and $\gamma$ 4. In this test the agent has run 17992 iterations. Here we try with different values in the learning algorithm, to see what would fit the best and which values the agent learns or converges the fastest.


%Damage given and taken
\insertmarginfigure{height=3in}{learningrate/A6G4/damage.png}
			{Alpha 6 gamma 4 Damage - Yellow: Damage given - Red: Damage taken}{fig:a6g4_damage}{-3in}

In figure \ref{fig:a6g4_damage} just as before but with different $\alpha$ and $\gamma$ values the agents graph over it's performance.


%Units killed and units lost

\insertmarginfigure{height=3in}{learningrate/A6G4/units_lost_and_killed.png}{Alpha 6 Gamma 4 Lost and killed - Yellow: Vultures left - Red: Marines killed}{fig:a6g4_lak}{-3in}



%%%%%AVERAGE%%%%%
\begin{centering}
 \begin{tabular}{|l||c|c|c|}
	\multicolumn{4}{c}{Average results from test 1.2} \\
	\hline
		Damage taken & Damage given & Units lost & Enemies killed\\
	\hline
		392,35 & 334,26 & 4,79 & 7,44 \\
	\hline
\label{test1.2}
\end{tabular}
\end{centering}
%%%AVERAGE%%%





Comparing the tests one can see that the agent needs several more iterations than it has right now, since it's not converging from just 40000 iterations. The next tests will be with around 100000 iterations and by that we can be more sure that it's near converging. The average values are almost the same, and what that tells us is that either it needs way more iterations or the small change in the values does not give a huge difference.




%%%% TEST WITH A4 G6 %%%%
\subsection*{Learning rate test 1.3}
In this test we are using the values $\alpha$ 4 and $\gamma$ 6. In this test the agent has run 135936 iterations.


%Damage given and taken
\insertmarginfigure{height=3in}{learningrate/A4G6/damage.png}
			{Alpha 4 gamma 6 Damage - Yellow: Damage given - Red: Damage taken}{fig:a4g6_damage}{-3in}

In figure \ref{fig:a4g6_damage} just as before but with different $\alpha$ and $\gamma$ values the agents graph over it's performance.


%Units killed and units lost

\insertmarginfigure{height=3in}{learningrate/A4G6/units_lost_and_killed.png}{Alpha 4 Gamma 6 Lost and killed - Yellow: Vultures left - Red: Marines killed}{fig:a4g6_lak}{-3in}

%%%%%AVERAGE%%%%%
\begin{centering}
 \begin{tabular}{|l||c|c|c|}
	\multicolumn{4}{c}{Average results from test 1.3} \\
	\hline
		Damage taken & Damage given & Units lost & Enemies killed\\
	\hline
		395,51 & 338,29 & 4,86 & 7,52 \\
	\hline
\label{test1.3}
\end{tabular}
\end{centering}
%%%AVERAGE%%%




%%%% TEST WITH A2 G9 %%%%
\subsection*{Learning rate test 1.4}
In this test we are using the values $\alpha$ 2 and $\gamma$ 9. In this test the agent has run 30852 iterations.


%Damage given and taken
\insertmarginfigure{height=3in}{learningrate/A2G9/damage.png}
			{Alpha 2 gamma 9 Damage - Yellow: Damage given - Red: Damage taken}{fig:a2g9_damage}{-3in}

In figure \ref{fig:a2g9_damage} just as before but with different $\alpha$ and $\gamma$ values the agents graph over it's performance.


%Units killed and units lost

\insertmarginfigure{height=3in}{learningrate/A2G9/units_lost_and_killed.png}{Alpha 2 Gamma 9 Lost and killed - Yellow: Vultures left - Red: Marines killed}{fig:a2g9_lak}{-3in}

%%%%%AVERAGE%%%%%
\begin{centering}
 \begin{tabular}{|l||c|c|c|}
	\multicolumn{4}{c}{Average results from test 1.4} \\
	\hline
		Damage taken & Damage given & Units lost & Enemies killed\\
	\hline
		398,78 & 282,28 & 4,96 & 6,31 \\
	\hline
\label{test1.4}
\end{tabular}
\end{centering}
%%%AVERAGE%%%


Comparing tests \ref{test1.4} and \ref{test1.3} where the 1.3 test has run 135936 iterations one can clearly see that the average damage dealt is higher than from the test 1.4 where the iterations are 30852 times. To make the agent converge is has to run even more iterations. If we cut the test \ref{test1.3} down to 30852 iterations and compare the numbers again it has an average of \textit{Damage taken: 396,05 - Damage given: 324,81} which is closely to the tests with few iterations.