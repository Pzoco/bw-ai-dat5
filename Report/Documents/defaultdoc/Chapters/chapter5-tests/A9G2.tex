
%%%% TEST WITH A9 G2 %%%%
\subsection*{Learning rate test 1.1}
In this test we are using the values $\alpha$ 9 and $\gamma$ 2 and the agent has run 40182 iterations


\textit{The figures showing in the test, can be watched full sized in the appendix \ref{appendix}} 
%Damage given and taken
\insertmarginfigure{height=3in}{learningrate/A9G2/damage.png}
			{Alpha 9 Gamma 2 Damage - Yellow: Damage given - Red: Damage taken}{fig:a9g2_damage}{-3in}

In figure \ref{fig:a9g2_damage} one can see every time the y-axis peaks (yellow graph) the agent have killed all of the opponent as we have hoped for. The red graph is damage taken, if it's on 400 all the 5 vultures have died and the agent have had a lost.

%%%%%AVERAGE%%%%%
\begin{centering}
 \begin{tabular}{|l||c|c|c|}
	\multicolumn{4}{c}{Average results from test 1.1} \\
	\hline
		Damage taken & Damage given & Units lost & Enemies killed\\
	\hline
		392,71 & 337,26 & 4,79 & 7,51 \\
	\hline
\label{test1.1}
\end{tabular}
\end{centering}

%%%AVERAGE%%%





%Units killed and units lost
\insertmarginfigure{height=3in}{learningrate/A9G2/units_lost_and_killed.png}
			{Alpha 9 Gamma 2 Lost and killed - Yellow: Enemies killed - Red: Vultures left}{fig:a9g2_lak}{-3in}
In figure \ref{fig:a9g2_lak} the yellow graph is how many marines was killed, and when it peaks to the 12 mark our agent have won a battle

