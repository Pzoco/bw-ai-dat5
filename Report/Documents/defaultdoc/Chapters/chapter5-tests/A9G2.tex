%%%% TEST WITH A9 G2 %%%%
\subsection*{Learning rate test 1.1}
In this test we are using the values $\alpha$ 9 and $\gamma$ 2 and the agent has run 40182 iterations


\textit{The figures showing in the test, can be watched full sized in the appendix \ref{appendix}} 

%Damage given and taken
\insertmarginfigure{height=3in}{learningrate/A9G2/damage.png}
			{Alpha 9 Gamma 2 Damage - Yellow: Damage given - Red: Damage taken}{fig:a9g2_damage}{-3in}

%Units killed and units lost
\insertmarginfigure{height=3in}{learningrate/A9G2/units_lost_and_killed.png}
			{Alpha 9 Gamma 2 Lost and killed - Yellow: Enemies killed - Red: Vultures left}{fig:a9g2_lak}{-3in}
In figure \ref{fig:a9g2_lak} the yellow graph is how many marines was killed, and when it peaks to the 12 mark our agent have won a battle




In figure \ref{fig:a9g2_damage} one can see every time the y-axis peaks (yellow graph) the agent have killed all of the opponent as we have hoped for. The red graph is damage taken, if it's on 400 all the 5 vultures have died and the agent have had a lost.

%%%%%AVERAGE%%%%%
\begin{centering}
 \begin{tabular}{|l||c|c|c|}
	\multicolumn{4}{c}{Average results from test 1.1} \\
	\hline
		Damage taken & Damage given & Units lost & Enemies killed\\
	\hline
		392,71 & 337,26 & 4,79 & 7,51 \\
	\hline
\end{tabular}
\end{centering}
%%%AVERAGE%%%

Comparing the potential field data in winning streaks could tell us how the agent reacts when it wins. Damage taken and damage given are self-explanatory, but the rest are all values of the potential fields. Ally is how far away each of our own vultures can be from each other, a negative value mean repulsive and a positive value mean attractive. The squad value is similar to the ally value but the squad means the center point in the group of units. The maximum distance is the distance to an enemy, the value is only for the potential fields and not real distance to the enemies. Cooldown is when the vultures have fired they use the cooldown value if they should be attracted to the enemy or not and lastly the edges are how attractive or repulsed the vultures are to the edges, and just as before, negative values means repulse and positive values mean attractive. 
\newpage
%%Comparing the winning streak%%%
\begin{centering}
 \begin{tabular}{|c||c|c|c|c|c|c|}
	\multicolumn{4}{c}{Winning streaks 1.1} \\
	\hline
	Damage taken & Damage given & Ally & Squad & Maximum distance & Cooldown & Edge \\
	\hline
	
	252&		480&			910558&	444338&	-306553&			316093&	432800\\
	400&		240&			910605&	444187&	-305894&			316093&	433077\\
	326&		480&			910560&	443930&	-306492&			316093&	432797\\
	400&		380&			910425&	443572&	-307812&			316093&	431980\\
	258&		480&			910220&	443116&	-309341&			316093&	430732\\
	292&		480&			909916&	442564&	-310620&			315902&	428863\\
	374&		480&			909862&	442487&	-310731&			315761&	428523\\
	312&		480&			910054&	442688&	-310426&			316235&	429738\\
	374&		480&			910113&	442768&	-310431&			316379&	430099\\
	400&		320&			910320&	442768&	-310431&			316935&	431369\\
	306&		480&			910355&	442491&	-310431&			316812&	431296\\
	400&		380&			910380&	442491&	-310431&			316852&	431531\\
	400&		300&			910352&	442491&	-310431&			316546&	430904\\
	392&		480&			910352&	442491&	-310431&			316085&	429879\\
	400&		420&			910352&	442491&	-310431&			315663&	428999\\
	312&		480&			910618&	443021&	-310431&			315663&	428999\\
	338&		480&			910636&	443043&	-310159&			315663&	428999\\
	362&		480&			911106&	443531&	-301755&			315663&	428999\\
	306&		480&			911550&	444136&	-291521&			315663&	428999\\
	\hline
	\label{winning_streak_1.1}
\end{tabular}
\end{centering}
The table \ref{winning_streak_1.1} shows some numbers when the agent have had a winning streak. One can see that the damage given are very high and the ones with 480 in damage given is when the vultures have won over the 12 marines. The values have been saved from the agent in the iterations 39484 to 39502. By looking at the numbers in ally and squad we can see that the units like to stick to each other, in other words they like to team up. The reinforcement learning have changed the values and it have found out that sticking together is better than dealing with a bunch of marines on it's own. The maximum distance, which means the attractive level towards the marines, are very high in a negative value which makes them very attracted towards the marines and in other words very aggressive. 



%%Comparing the winning streak%%%