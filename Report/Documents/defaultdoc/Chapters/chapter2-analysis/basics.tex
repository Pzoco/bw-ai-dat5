Before we actually can make a bot, we have to define how the game works and how a player can win.
In this chapter we will define the basic rules of the game, the ways a player can win in abstract sense and in a concrete way.

\section{Basics}
	Starcraft is a complex game in which a player puts his army against his opponent's army in order to beat him. 
	There are many complex strategies involved in doing this, but ultimately this game can be described in simple terms.\\
	Before the match starts you get to choose between three different races: Protoss, Terran, or Zerg. 
	Even though all of these races are unique with different strategies and units, the basics of the game remain the same no matter which race you choose.\\
	The first rule of the game is that if your last building is destroyed then you lose. 
	This means that in order to win against an opponent you must kill his last building before he kills your last building. \\
	At the beginning of the game you start out with one building that can only train a most basic unit which can collect resources. 
	Resources are used to buy buildings, units, and upgrades. 
	So it is important to collect more resources than your opponent so you can buy more buildings, units, and upgrades than your opponent.
	This can be a difficult task. It is not enough just to get more resources than your opponent. 
	You need to utilize your resources. Resources that are not being used for anything are wasted. 
	In general you want to have a high income but keep a low amount of resources in reserve. 
	So if you have extra resources you need to either buy more buildings, train more units, or buy and upgrade for your army.\\
	In order to destroy your opponents building you must have an army (or at least some units). 
	In the beginning of the game you have almost no army and you want to make your army stronger as the game progresses. 
	It is not good enough to just have an army, but you must have a better army than your opponent. 
	Better does not always mean that your army can kill an opponent's army in a confrontation. 
	Your army just needs to keep you alive in a way to put you ahead of your opponent.\\
	Generally there is a trade off between your army and your economy. 
	The larger the army you have the less you are spending on growing your economy (and vice versa). 
	If you spend too much on your army but not enough on economy you may become too far behind later in the game to win. 
	On the other hand if you spend all of your resources on your economy the enemy army could come and wipe you out.\\

 