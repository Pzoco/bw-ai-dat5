\section{Important Techniques}
%\section{Aspects of winning}
In this section we will introduce the elements of Starcraft Broodwar which have to be taken into account in order to win a match. This will give a understanding 
of Broodwar and will later be useful in making our bot. We will take a look at build orders, information gathering, macromanagement and
micromanagement.

	\subsection{Build Orders}
		A build order is an early game strategy describing in which order certain building and units should be produced\cite{wiki_build_order}. 
		The purpose of a build order is to optimize the player's economy and to control the timing of units which can give 
		the player advantages at certain points in the game.\\
		The build order depends on several factors like the specific map played and the race of the opponent. 
		Early game scouting can help determine the opponent's build order and so you can counter their strategy.\\
		
		A build order is most important at the beginning of the game and relies a lot on timing. 
		The player must choose a spot to build a certain building, ensure a worker is there to do it, 
		and make sure the player has the required resources. 
		A good player knows the exact time to do these kinds of actions. 
		He will make worker reach the construction site at the exact time he gets enough minerals to actually build the building.\\
		
		Later in the the game the player must adapt his strategy depending on the information received about the opponent.
	
	\subsection{Information Gathering}
		Every strategy and decision made throughout the game depends on the pieces of information 
		known in the current moment. From the most basic decisions like where to build, what to build, and
		which units to build; to wider decisions like how to defend, when to attack, when to rush, or how to counter-attack.
		Everything depends on the known information. The decision making process begins and grows with the input information.
		How the agent obtains pieces of information is just as important 
		as how it process them. You do not want to waste too many resources finding out information otherwise you will waste units. 
	
			\subsubsection{Map Knowledge}
				During the game many of the actions have to rely the map that the player's are playing on. Movements, starting point, 
				tactics, and general strategy depend on how much data there is about the map. 				
				For example the resources you need for building and expanding through the game are normally distributed
				equally between the players, and the players are normally placed at opposite ends of the maps. 
				Using this general map information and the information obtained in the moment the game starts 
				(map, size, exact location, opposite race), a player will decide on his build order and starting actions. 
		
			\subsubsection{Scouting}
				Reasons for exploring the map are not just to find out about the geographical structure, 
				but to obtain information on the actions of the opposing player. The player's actions will depend on
				the enemy's position, buildings, units, tactics and strategy. The processing of the information 
				obtained from scouting is important and delicate since it is obtaining small pieces of information 
				that structure a large network of unknown information. Also, this other source of information 
				is more volatile and temporal than the previous considerations because it is only obtained for the 
				limited amount of time the unit stays in a certain place and to the areas the buildings limit the fog of war.
				
	\subsection{Macromanagement}
		Macromanagement often just called Macro is a key concept in Starcraft Broodwar. 
		Macro is the ability to use all ones resources and the ability to expand at the right times to keep a healthy economy.
		The most difficult part of macro is to keep up production of units and keeping down resources while attacking.
		If players focus on economy and don't attack each other, the game is considered a macro game. This defensive tactic where a
		player stays in his base and macro up a big army is called turtling. Macro heavy games, where players are turtling, lead to
		long games with massive battles. 
			
			\subsubsection{Queuing}
				If a player has good macro he should not queue up units. 
				Every time you queue a unit the resources are withdrawn instantly before the unit begins its production time. This means that optimally you only want to have one unit in a queue at a time for each producing facility.
				This is also the case with workers. Normally the player wants to keep building workers throughout the entire match in order the have the best economy.
				In some cases it can be an advantage to stop building workers and focus on spending resources on something else like a building.
				
				
	\subsection{Micromanagement}
		Micromanagement also known as micro requires a player to control their units so that you can get the most out of them.\\
		
		For a human player this can be very difficult as it requires a lot of concentration and can hurt an inexperienced players macro. 
		An advantage for a computer player is it does not have to balance its concentration between micromanagement and macromanagement.\\
		
		One way to do micro is to keep your units alive as long as possible. 
		If your units stay alive after a battle then you do not have to replace as many units as your opponent. 
		Players can move hurt units out of battle and then back in. 
		The reason for this is that the enemy will then start attacking another one of your 
		units and then your hurt unit can enter back in the battle and continue doing damage. 
		A unit will do the same amount of damage no matter how much health it has but will do no damage if it is dead.\\
		
		Another form of micro is making several of your units attack one unit at a time, this is called \textit{focus firing}.
		A computer can do this very well. Computers can perform calculations to find how many units it takes to kill another unit in one shot.
		This is useful so that too many units don't waste their shots in killing one unit when they can be doing damage to other units.\\
		In order to micro units effectively you must be able to do a lot of different actions at one time. 
		This is generally measured by a unit called \abapm. 
		A computer can have an extremely high \abapm. Making it easy for it to micro its units precisely.

\subsection{Meta-Game}		
%	\subsection{Unpredictability}
		When playing tournaments or even just playing a normal match, being unpredictable can win you the game. 
		If your opponent is adapting to your play style you can throw him off by doing something different or playing strange.
		In tournaments psychology is a big factor. Professional players can make the other player do a certain strategy because they know 
		how the other player will react to certain things. This feature will be hard to near impossible for a bot to learn because the bot won't be able to 
		identify the opponent's playstyle in depth. Because the human can do a creative new build that the bot don't know and can't react to. 
		If it can not identify the playstyle it will not be able to put the opponent off.