\chapter{Design}\label{design}
	In this chapter we will distinguish between the word agent and bot, where an agent is a program thats able to learn and chance itself, and the bot it the more static. \\
	
	In this chapter we are going to look at the design of the bot. First we will take a look at the different managers we use for controlling units, producing buildings, scouting bases and choosing strategies. Then we explain potential fields, and how we use them to micro the bot. In order to correct the bot and make sure that it will be able to learn, we will explain reinforcement learning. For the macro part we are going to use Bayesian networks to detect the opponent's build order, predict the opponent's start location, and find the threat level of the opponent.	\\
	The purpose of the design is to make an agent that is able to out micro its opponent in combat. To limit the scope we will only look at vultures. 
	This implies the following:
	\begin{itemize}
		\item Construct the necessary buildings to produce vultures.
		\item Construct a squad of vultures.
		\item Scout and find the enemy's base.
		\item Move the squad to the enemy's base and attack.
	\end{itemize}
	
%	Potential fields are a vector based representation of an environment. Potential fields are commonly used in robotics to guide a robot trough environment, but it is also becoming more and more popular within computer games\ref{ptf_in_game}. Objects in this environment can change position and the field will adapt accordingly. 
	\section{Bot Managers}
	\label{design::managers}
	The managers for the bot are used to control different parts of the bot. The part which is connected to the micromanaging of units will use 
	reinforcement learning and Potential fields. The part which manages scouting information will use Bayesian networks. 
	The purpose for these managers are to separate parts of the game as much as possible. They should only use the other managers when a task needs to be handled 
	by another manager. The following subsections will explain the different managers.
	
	\subsection*{Tactics Manager}
		This manager is responsible for managing our offensive units in and out of small scale combat. 
		It manages the units by controlling squads of different units.
	\subsection*{Production Manager}
		This manager is responsible for producing units, constructing buildings, upgrading technology and researching technology. The manager will be able to 
		know what to build by following build orders.
	\subsection*{Scouting Manager}
		The Scouting Manager keeps track of the opponent. 
		It saves every enemy it has seen and is responsible for scouting the enemy's spawn position. 
		It also keeps track of which buildings the enemy has produced.
	\subsection*{Strategy Manager}
		The Strategy Manager is responsible for making high-level decisions based on the information we have. These decisions include
		setting the Tactics Manager to attack the enemy or setting the Production Manager to change build order.
	\subsection*{Worker Manager}
		The Worker Manager sets the workers to mine minerals, mine gas, scout, construct buildings and perform other tasks.
	\subsection{Managers in Game Scenario}
		
		The game starts and the Production Manager begins following the build order which produces SCVs. Every time an SCV is created the 
		Worker Manager sets the worker to a free mineral patch. The Scouting Manager sends out an SCV to the opponent's base. The scout 
		finds the base and the Strategy Manager still predicts that the threat level is low. The Production Manager keeps following the build order, and now 
		we have marines assigned to a Tactics Manager. The Scout Manager sends out another SCV, but this one is killed before reaching the base of the 
		opponent. The Strategy Manager sets the threat level to high and makes the Production Manager stop producing workers. The production manager will then start producing only offensive units. 
		The opponent does the attack, and the Tactics Manager micromanages the units which defeats the enemies.

	\section{Potential fields}
	Potential fields can be used to represent an area of the game using vectors.  The vectors are calculated based on all the objects in the area, and then used to calculate a path trough the area.

	Objects can have an attractive or repulsive type of influence in the potential field.
	
If we begin by introducing a single enemy unit into an empty out world the entire area begin flooded with vectors pointing toward this point, as seen in figure \ref{fig:seekbehavior}. This is called the Attractive behaviour because it makes our own units want to reach this point in order to attack it.
	
	\insertmarginfigure{height=3in}{Potentialfields/seek.png}
		{Attractive behavior\cite{pft}}{fig:seekbehavior}{-3in}
	
	Likewise if we assume there is only a single obstacles in the area (a unit we do not want to attack) it would generate a repulsive field around it, see figure \ref{fig:avoidbehavior}. This is called the Repulsive behavior because it causes our own units to try and move away from it.

	\insertmarginfigure{height=3in}{Potentialfields/avoid.png}
		{Repulsive behavior\cite{pft}}{fig:avoidbehavior}{-3in}
		
	These two kind of behaviours can then be combined to make a map that can tell our unit how to move around enemy units and reach a specific target as seen in figure \ref{fig:combinedbehavior}.
	
	\insertmarginfigure{height=3in}{Potentialfields/combined.png}
		{Combined behavior\cite{pft}}{fig:combinedbehavior}{-3in}
		
	\subsection{The math behind potential fields}
		A simple potential field can be described as below.\\
		
		Variables:\\
		$f =$ force\\
		$s =$ size of the potential field\\
		$c =$ fade of constant\\
		$d =$ distance\\
		\begin{displaymath}
			%\begin{math}
			Attractive = \begin{cases}
					-f & \text{if $d < s$}\\
					0 & \text{else}
				\end{cases}		
			%\end{math}
		\end{displaymath}
			
		\begin{displaymath}
			%\begin{math}
			Repulsive = \begin{cases}
					0 & \text{if $d > s$}\\
					f * c & \text{else}
				\end{cases}		
			%\end{math}
		\end{displaymath}
		But we need to have more then just these to simple cases as explained in the next chapter.
		
		
	\section{Parameters for the potential fields}
	This section will contain the parameters, which will be tweaked in order to give the vulture an optimal behaviour. Because we are tweaking 
	potential fields the parameters will either be repulsive or attractive.
	
	\subsection*{Squad center (Attractive)}
		A vulture is more likely to survive if it sticks together with it's squad. It will both give a better damage output and also spread out the 
		damage taken.
		\begin{displaymath}
			TODO = \begin{cases}
					a & \text{if}\\
					b & \text{else}
				\end{cases}		
		\end{displaymath}
		
		
	\subsection*{Maximum distance positioning (Attractive)}
		The vultures have to utilize the range they have and therefore they will be attracted to the position which gives them the maximum distance 
		to the enemies.
		\begin{displaymath}
			TODO = \begin{cases}
					a & \text{if}\\
					b & \text{else}
				\end{cases}		
		\end{displaymath}
	\subsection*{Ally units (Repulsive)}
		Because the vultures are not supposed to clump up, we will add some repulsion to ally units.
		\begin{displaymath}
			TODO = \begin{cases}
					a & \text{if}\\
					b & \text{else}
				\end{cases}		
		\end{displaymath}
	\subsection*{Weapon cool down (Repulsive)}
		The vultures is only needed in range of the enemies if they can shoot, so some repulsion is added when attack cool down is present.
		\begin{displaymath}
			TODO = \begin{cases}
					a & \text{if}\\
					b & \text{else}
				\end{cases}		
		\end{displaymath}
	\subsection*{Edges and cliffs (Repulsive)}
		As with the ally units, we do not want the vultures to clump up against the wall so cliffs and edges will be repulsive.
		\begin{displaymath}
			TODO = \begin{cases}
					a & \text{if}\\
					b & \text{else}
				\end{cases}		
		\end{displaymath}
	
	
	%%\section{Conclusion on Potential fields}
%	Potential fields consist of several smaller fields, each with it's own variables that can be adjusted to optimise the Potential field as a whole. Each of these variable can then be taught using some kind of Machine Intelligence. It is also easy to add new or remove existing fields if needed, making the Potential field highly adjustable which is a good feature in a game where small errors can have a large impact.

	\section{Reinforcement Learning}

	Reinforcement learning is a method used to build a model or functions that learn from experience and examples. The basic idea is that for every action in an environment there is a reward or feedback that reinforces all the actions that have bigger rewards. In a bigger scale the task of reinforcement learning is the process to discover the optimal path or series of actions to accomplish the best possible reward at the end of the process. 

	The choice of using reinforcement learning was based on the necessity to train our AI to perform better after every match or test. Since our environment covers a lot of different factors and variables, we then decided on using a form of active reinforcement learning that simplifies the complexity and size of all the different states in the environment: Generalization. 

	Generalization in reinforcement learning takes into consideration huge state spaces by representing them by function approximation. This function reduces the complexity of mapping all the states considerably and allows the learning agent to generalize form the visited states to the not visited ones. 

	We combine the reinforcement learning method with the potential fields by transforming all the potential fields of each unit into the representation of the utility function used by the agent. We do this by talking all the forces that determine the magnitude of the potential field vectors as coefficients in the utility function:
	
	
	Reinforcement learning can be used in two different ways, either an active or a passive way. By using the passive reinforcement learning one defines a specific policy of the agent, and the task is to learn the utilities of states \cite{rl}. The basic goal is to learn how good the predefined policy is, and it does that by iterative going through the states without knowing the rewards beforehand and by trying and 'remembering' the best way. The active reinforcement learning does not have a fixed policy to begin with, and the agent must decide what actions to take \cite[p771]{rl} by that in mind we have to learn an agent a complete model with what outcome for every action. The basic idea behind active reinforcement learning is that in the end it should be taught and obeying the optimal policy. It learns the utility function 

	% There should be something about:
		% Passive Reinforcement / Active Reinforcement
			% Direct utility estimation
			% Adaptive dynamic programming
			% Temporal difference learning
			% Exploration
			% Q-Learning
			
%write about NEURAL NETWORK, why we don't use it.





\subsection{Neural Network}			 
			

%Intro
Neural network [NN] can be used to teach an agent to behave a certain way.
A neural network is similar to how the brain obtains new information, and learns it. The human brain is made up of about 100 billion neurons, and each neuron is connected with thousand of other neurons communicating via electrochemical signals \cite{nn}. The neuron gets a lot of input and when the neuron has filled a threshold it fires through something called an axon. This automatic method of filling up information and storing it when a certain threshold is reached can be implemented in a computer.



%Description
There are different ways of connection a neural network, but we will be concentrating on the method called \textit{feedforward}. The way to feed the neuron is by applying different inputs to it, and before entering the neuron the input goes through some weights, that is adjusted manually. Just like before if the input in total is greater or equal to a certain threshold the neuron will output a signal, if it's less it will output zero.
%Technical details
A neuron can have from 1 to n number of inputs, and each input will be multiplied by its weight
If we have input $x_1, x_2, x_3, x_n$ the weights calculated will look like $x_1 w_1 + x_2 w_2 + x_3 w_3 + .. +  x_n w_n$

%Why we're not using NN

We have chosen not to use neural network and the main reason is that neural network is very slow, and in a real-time-strategy game where we can have around 50.000 states, the computation time will be very slow and the input will be way to high every minute it has to compute approximately 50.000 states. A generalization function would be a better approach because it approximates the states, and doesn't have to compute the entire state-space in a game. 
%50.000 states because APM Multiplied with TimeSpent in the game 

	\section{Bayesian Networks and Decision Trees}
In this section we are going to define Bayesian networks and decision threes. Then we are going to compare the two decision models and choose
the best model our bot can use for scouting analysis.

\subsection{Bayesian Networks}
	Bayesian networks are simple graphical models, where each probability for the variable is calculated. Therefore the Bayesian networks is used for
	calculating new probabilities, whenever new information is gathered. So a Bayesian networks have a  set of variable and
	these variable is connected with directed edges. Each variable in the Bayesian network must have a finite set of mutually exclusive states. By this we	
	mean that :::. To make sure that we can calculator a result, the Bayesian network needs to be a acyclic directed graph. A variable needs a conditional 		
	probability	table for each of it's

\subsection{Decision Trees} parents. This means that the amount of calculations in a Bayesian networks depends on the number of variables and the 		edges that connects them.
	Decision trees are used to represent decision problems. A decision tree consists of three types of nodes: decision nodes, chance nodes and utility nodes. 
	The link from a decision node to a chance node is called an action, and a link from a chance node to a decision node is called a state. 
	The idea of the decision tree is to find the path that will give us the highest utility (reward), so to make a decision tree over a decision problem, 
	every possible path of decisions have to be shown in the tree. This will give a tree that grows exponentially with the number of decision and 
	chance variables, so given small decision problems will require big trees, if not reduced. 
	
	A decision tree could be made for predicting the enemy's spawn location or their strategy. 
	
\subsection{Conclusion}

	\section{Designing Bayesian Networks for Prediction}\label{bayesian_network}
This section will describe the Bayesian networks we made to predict the enemy spawn position, enemy build order, and enemy threat level. The next subsection describes the network created for predicting the enemy spawn position.

\subsection{Prediction of Enemy Spawn Position}			 
			
The purpose of this Bayesian network is to find our opponent's spawning position. Some factors taken into account are our spawn position, enemy scout positions, and time to help us predict this. We wanted to make this network usable on a variety of maps. Below is a visual representation of how each node is connected in the network.
%Why do we represent time as almost none, early, middle, and late?
%actual time for these times - refrace.

\begin{figure}[H]
	\includegraphics{Figures/BayesianPictures/SpawnPrediction.png}
	\caption{A graph of the nodes in the Spawn Prediction Bayesian network}
	\label{fig:predicting}
\end{figure}

\subsubsection*{EnemySpawn}
The correct state in this node is the value we are trying to discover. Ultimately, all of the evidence we collect, will be used to find this state. Most of the other nodes have direct links from this node. Once we know the state of this, we are satisfied and do not need to use the network to infer more information because we do not care about predicting any of the other states.

\subsubsection*{OurSpawn}
This node is a child of EnemySpawn. We can use the values in this node to determine where the enemy does not spawn. We simply say that the enemy cannot spawn where we spawn at. At the beginning of a match we know where we spawn and can instantly put evidence on the OurSpawn node. This instantly reduces the change for our opponent to be at any other spawning location to 33\%.

\subsubsection*{EnemyNotAt(NE,SE,SW,NW)}
These nodes are used to keep track of positions we know our opponent did not spawn. Since a node can only be in one state at a time, we made four nodes that can influence the probabilities in enemy spawn. Ways we would generally get these values could be our own scouts arriving at a position and not finding our opponent.

\subsubsection*{WorkerScoutPosition} When we observe an enemy worker at a certain position on the map, we can influence our belief on our opponents spawn position. This is because the opponent's scout will take a certain amount of time to get to a certain position on the map. TimingSeen is a parent of this node. We use the information from this node to help our prediction. Just seeing an enemy at a certain position isn't enough. We need the time we see the enemy scout to help us form our beliefs. Whenever we obtain evidence on WorkerScoutPosition we will also always gather information on TimingSeen.

\subsubsection*{TimingSeen} This node is used to help us use the information from WorkerScoutPosition. It has the values: Almost None, Early, Middle, and Late. These values are the times we may see the opponent's scout. Since the values for timing depend on both map size and map layout, we purposely make the values broad.


\subsubsection*{OverlordDirection} Since overlords are so slow, a player will be able to use the direction the overlord is coming from to predict the spawn location. By the time an overlord would have visited two bases we probably would have already known where the enemy's base is. This node only helps when fighting against Zerg players. EnemySpawn is a parent of OverlordDirection. Once we know the direction the overlord is coming from we gain a lot of information on the enemy's spawn position.

\subsection{Build Order Prediction}
	This subsection describes the Bayesian networks that were created for predicting which build order the enemy is doing. 
	The networks have some similarities between them. There are three different networks, one each for every race that the bot can encounter: 
	Terran versus Terran, Terran versus Protoss and Terran vs Zerg.  \\
	
	They all have a node called BuildChosen, which is the node containing the beliefs for what the enemy's build order is. 
	The other nodes are buildings (Except an upgrade node in the Terran vs Protoss network) all with the states Seen and NotSeen. Seen is only used when presenting evidence because the bot will never be able to prove that the enemy building does not exist. 
	These states determine if we have scouted the building. 
	The order in which the buildings appear in the network is modelled with arrows going from the nodes representing the prerequisite buildings needed to the node. 
	As buildings are scouted evidence is put on the state Seen. The nodes and the probabilities for the most probable build order is then increased.

\subsubsection{Terran versus Terran}
	This network will try to predict the following build orders: Proxy Rush, 2 Factory Vulture Pressure, 1 Factory Expand, 1 Starport Wraith and Stim Rush. 
	The numbers in the node names are how many of the given unit the enemy has, e.g. Barracks2 means the second barracks the bot scouts. 
	
\begin{figure}[H]
	\includegraphics{Figures/BayesianPictures/tvt.png}
	\caption{Terran versus Terran prediction network}
	\label{fig:tvtnetwork}
\end{figure}			
\subsubsection{Terran versus Protoss}
	This network is larger than the other two prediction networks. Instead of only having a node for BuildChosen it has a node for the opening 
	used. The reason for this is that the Protoss are a very versatile race, and the different openings effect when the build order hits and what kind of build 
	the player is most likely to use. After the opening is determined all the nodes affecting the openings are not updated any more. The build orders 
	the network can predict are : 2 Base Carrier, 2 Base Reaver and 2 Base Arbiter. The openings the bot can predict are: One Gate Tech, Two Gate Range, 
	Nexus First and Two Gate Zealot Rush.

\begin{figure}[H]
	\includegraphics[scale=0.7]{Figures/BayesianPictures/tvp.png}
	\caption{Terran versus Protoss prediction network}
	\label{fig:tvpnetwork}
\end{figure}	

\subsubsection{Terran vs Zerg}
	This network will try to predict the following build orders: 2 Hatch Muta, 3 Hatch Muta, 3 Hatch Lurker and Early Pool Rush. Early Pool Rush is a 4pool or 5pool. The nodes have names similar to the Terran network where a node's name ends on the number of that type is scouted. 
	E.g Hatchery3 means that the bot have scouted 3 hatcheries.

\begin{figure}[H]
	\includegraphics[scale=1]{Figures/BayesianPictures/tvz.png}
	\caption{Terran versus Zerg prediction network}
	\label{fig:tvznetwork}
\end{figure}	
		
			
\subsection{ThreatLevel Prediction}
	This prediction is closely related to the prediction of the enemy build order. A build order has a certain time where it is effective 
	or where it hits. To determine what the current threat level is two nodes have to be added to each of the build order prediction networks, 
	ThreatLevel and Time. ThreatLevel has the states Low, Medium, and High. To make the network simple it only has three states for time: 0-5 minutes, 
	6-8 minutes and 9-11 minutes. By presenting evidence like before and putting evidence on the time node, the current threatlevel can be read.
\begin{figure}[H]
	\includegraphics[scale=1]{Figures/BayesianPictures/threatlevel.png}
	\caption{ThreatLevel for the Terran vs Terran matchup}
	\label{fig:threatlevel}
\end{figure}	










 

	\section{Design Conclusion}

Test Design Conclusion
