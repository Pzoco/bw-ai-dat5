In this Chapter we will describe the test results of mirco battles to test our AI. We have two test maps, which all our test are going to be performed in. The first map is  zerglings against vultures and the other map is marines against vultures. We use the first map with zerglings because zerglings have a lower movement speed than vultures, so the vultures should be able to win the battle without any units lost. The second map is more interesting because the marines are ranged units, this means we can see how well our bot do compared with the built-in bot. The maps are  designed so that the player with the vultures are weaker than the zerglings on the one map or the marines on the other. This means our bot can improve and should be better than the built-in bot. In the first map it is 20 marines against 5 vultures, which should be a win to the marines if they were controlled by a human, and the other map is 30 zerglings against 5 vultures. Each test will be performed on both maps. \\


In the first test we will make the build-in AI fight itself, to know how the vultures are doing, and have a base case to compare with. This will help us understand how strong the two armies are compared with each other and what losses we can expect. The other test will test the potential fields and what effect they have on the result.
In order to test this we will look at the difference with potential fields and without them. The last test we will test reinforcement learning on a map with 5 vultures against 12 marines, and see how fast the bot will improve over time. We will do this by comparing several test runs.

\section{First Test} % VvsZ without RL and PF
In this test the two forces are just attacking each other without any use of potential fields or reinforcement learning, this is without any micromanagement control. This means the vultures are losing in both maps, which they should so we can prove that if a unit is being controlled rightly a win is possible. Below is a list of the results from the test in the two maps.\\

\begin{centering}
 \begin{tabular}{|l||c|c|c|}
	\multicolumn{4}{c}{Test results from first map} \\
	\hline
		Players & Produced units & Killed units & Lost units\\
	\hline
	\hline
		Player with vultures & 5 & 9 & 5 \\
	\hline
		Player with zerglings & 30 & 9 & 5\\
	\hline

\end{tabular}
\end{centering}
\newpage
%VvsM

\begin{centering}
 \begin{tabular}{|l||c|c|c|}
	\multicolumn{4}{c}{Test results from second map} \\
	\hline
	Players & Produced units & Killed units & Lost units\\
	\hline
		Player with vultures & 5 & 5 & 5\\
	\hline
		Player with marines & 20 & 5 & 5\\
	\hline

\end{tabular}
\end{centering}

After this first test we know how weak the player with the vultures are compared to the opponent.
\newpage
\section{Second Test} %map is VvsZNew - with use of PF and NOT RL
This is the first test where we test our bot with only potential fields fighting the built-in bot. There are not used any reinforcement learning in this test.\\

\begin{centering}
 \begin{tabular}{|l||c|c|c|}
	\multicolumn{4}{c}{Test results from second map} \\
	\hline
	Players & Produced units & Killed units & Lost units\\
	\hline
	\hline
		Player with vultures & 5 & 30 & 0\\
	\hline
		Player with zerglings & 30 & 0 & 30\\
	\hline

\end{tabular}
\end{centering}

Out bot get this impressive result by not overextending itself and thereby only be in sight of a few zerglings. This forced some zerglings out of the group and made them a easy target. This test went better as expected that it took out some zerglings at a time and won the match with 5 vultures with full health.\\

\begin{centering}
 \begin{tabular}{|l||c|c|c|}
	\multicolumn{4}{c}{Test results from second map} \\
	\hline
	Players & Produced units & Killed units & Lost units\\
	\hline
	\hline
		Player with vultures & 5 & 6 & 5\\
	\hline
		Player with marines & 20 & 5 & 6\\
	\hline

\end{tabular}
\end{centering}


The result from this test wasn't as good as one would expect. The result of the vultures moving back and forth should be way better then static movement. But the fact is that the vultures only killed one more marine. The reason why this is the case is that the vultures couldn't use the same attack pattern as effective as with the zerglings, because the marines range attack sometimes dealt damage to the vultures. So the vultures is getting to close before attacking.
We see from this test that potential fields in itself is not enough, if we like to win or have less loss then the opponent. In the next test we will test if there is an improvement by using reinforcement learning on the bot.




