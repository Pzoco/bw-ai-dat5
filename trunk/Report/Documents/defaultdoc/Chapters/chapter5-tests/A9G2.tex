%%%% TEST WITH A9 G2 %%%%
\subsection*{Learning rate test 1.1}
In this test we are using the values $\alpha$ 0.9 and $\gamma$ 0.2 and the agent has run 40182 iterations


\textit{The figures showing in the test, can be watched full sized in the appendix \ref{appendix}} 

%Damage given and taken
\insertmarginfigure{height=3in}{learningrate/A9G2/damage.png}
			{Alpha 0.9 Gamma 0.2 Damage - Yellow: Damage given - Red: Damage taken}{fig:a9g2_damage}{-3in}

%Units killed and units lost
\insertmarginfigure{height=3in}{learningrate/A9G2/units_lost_and_killed.png}
			{Alpha 0.9 Gamma 0.2 Lost and killed - Yellow: Enemies killed - Red: Vultures left}{fig:a9g2_lak}{-3in}
In figure \ref{fig:a9g2_lak} the yellow graph is how many marines was killed, and when it peaks to the 12 mark, our agent have won a battle.




In figure \ref{fig:a9g2_damage} one can see every time the y-axis peaks (yellow graph) the agent have killed all of the opponent as we have hoped for. The red graph is damage taken, if it's on 400 all the 5 vultures have died and the agent have had a lost.

%%%%%AVERAGE%%%%%
\begin{centering}
 \begin{tabular}{|l|c|c|c|}
	\multicolumn{4}{c}{Average results from test 1.1} \\
	\hline
		Damage taken & Damage given & Units lost & Enemies killed\\
	\hline
		392,71 & 337,26 & 4,79 & 7,51 \\
	\hline
\end{tabular}
\end{centering}
%%%AVERAGE%%%

Comparing the potential field data in winning streaks could tell us how the agent reacts when it wins. Damage taken and damage given are self-explanatory, but the rest are all values of the potential fields. Ally is how far away each of our own vultures can be from each other, a negative value mean repulsive and a positive value mean attractive. The squad value is similar to the ally value but the squad means the center point in the group of units. The maximum distance is the distance to an enemy, the value is only for the potential fields and not real distance to the enemies. Cooldown is when the vultures have fired they use the cooldown value if they should be attracted to the enemy or not and lastly the edges are how attractive or repulsed the vultures are to the edges, and just as before, negative values means repulse and positive values mean attractive. 
\newpage
