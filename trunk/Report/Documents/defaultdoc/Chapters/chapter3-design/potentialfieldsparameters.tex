\section{Parameters for the potential fields}
	This section will contain the parameters, which will be tweaked in order to give the vulture an optimal behaviour. Because we are tweaking 
	potential fields the parameters will either be repulsive or attractive.
	
	Variables:\\
	$f_c =$ A force used to adjust the potential field.\\
	$da =$ distance to closed ally unit.\\
	$ds =$ distance from center of army to unit.\\
	$dsv =$ distance from center of army to current point.\\
	$sr =$ units maximum shooting range.\\
	$de =$ distance to enemy.\\
	$due =$ distance from current point to enemy.\\
	$wr =$ boolean denoting whether or not the weapons are ready to fire.\\
	$dc =$ distance to cliff or edge. \\
	$dua =$ distance from current point to enemy.
	
	\subsection*{Squad center (Attractive)}
		\label{SCA_label}A vulture is more likely to survive if it sticks together with it's squad. It will both give a better damage output and also spread out the 
		damage taken\footnote{Each of the cases in the potential fields will be explained in detail in the implementation}.
		\begin{displaymath}
			SquadCenter  = \begin{cases}
					0 & \text{if } dsv >= ds\\
					f_{S} \times (ds - dsv) & \text{if } ds > c\\
					0 & \text{if } ds \leq c
				\end{cases}		
		\end{displaymath}
		
		When we desined squad center, it became clear that we need three cases. 
		In case first we handle the situation where the current vector we are calculating is facing a way from the squad and need to be 0. If we didn't include this step the unit would also be attracted to tiles in the wrong direction. This is being done because the calculation is based on the distance to the squad and the tiles facing away is at a greater distance but should not be more attractive.
		
		Second case  is concerning the case where we are not within the desired range of the squad. So we calculate how strongly we want to move towards the squad, this depends on the distance. 
		
		Third case is all other cases, being that the unit is within the squad.
					
	\subsection*{Maximum distance positioning (Attractive)}
		The vultures have to utilize the range they have and therefore they will be attracted to the position which gives them the maximum distance 
		to the enemies.
		\begin{displaymath}
			MaximumDistancePositioning = \begin{cases}
					f_{MDP} \times (de - due + de) & \text{if } de > sr\\
					0 & \text{if } de = sr\\
					-f_{MDP} \times (de - due + de) & \text{if } de < sr
				\end{cases}		
		\end{displaymath}
		Maximum distance positioning contains to cases. The first is handling the situation where we are not witin the maximum range. In that case we cant to be attractet to the enemy. Note that in both cases we have $(due - de + due)$ which basicaly just invert the distances to the enemy so the closet tile will have the largest value.
		In the secound case we handle when we are exactly within the right distance, and just want to stay where we are.
		The last case takes care or the last option, we are to close and want to get away.
		We use to different forces to be able to adjust them indepentently later.
		
		
		
	\subsection*{Ally units (Repulsive)}
We do not want all of the vultures to clump completely together during a match. This can lead them vulnerable to spash damage attacks. If we keep the Vulture spread out a reasonable distance the spash damage will not take any effect.
		\begin{displaymath}
			AllyUnits = \begin{cases}
					0 & \text{if } da > c\\
					-f_{AU} \times da & \text{if } da \leq c\\
					-f_{AU} * (da - dua + da) & \text{if } da = 0
				\end{cases}		
		\end{displaymath}
	For the first case we wanted to make sure that if the vultures where far enought away from each other (the constant c) then they would not repulse each other at all. The reasoning behind this is simple because if the vultures are far away then they cannot take any splash damage.
	On the second case we want out units to repulse each other when they are close enough together. To do this we take our -fAU (force Ally Unit) and divide it by the distance to the ally unit. This way the closer our vultures get to each other the more they will want to move away.
	Our final case is if some how the vultures are right on top of each other or on the same position. Given our other equation this would cause a divide by zero error. So instead of using our second formula we use---------->insert more stuff
	\subsection*{Weapon cool down (Repulsive)}
A vulture cannot do any damage right after it shoots. This is because in there is a certain amount of time in between each shot. Right after a vulture shoots we want it to move out of the battle until it can fire again. This way each vulture will be less vulnerable to taking damage when it cannot give any damage.
		\begin{displaymath}
			WeaponCoolDown = \begin{cases}
					0 & \text{if } wr\\
					-f_{WCD} * (de - due + de) & \text{else}
				\end{cases}		
		\end{displaymath}
In our first case we give the field a value of zero if the vulture is ready to fire. We do not want it to retreat if it is available to fire.
The other case is if the vulture's weapon is not ready. In this case we take the distance from the enemy(due-de+due) and multiply it by the force.
	\subsection*{Edges and cliffs (Repulsive)}
		As with the ally units, we do not want the vultures to clump up against the wall so cliffs and edges will be repulsive.
		\begin{displaymath}
			EdgesAndCliffs = \begin{cases}
					\frac{-f_{EAC}}{de} & \text{if } de \leq c\\
					0 & \text{if } de > c
				\end{cases}		
		\end{displaymath}
	
	