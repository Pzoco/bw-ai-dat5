\section{Bayesian Network}\label{bayesian_network}

Here will be a description of how we have modelled our Bayesian network.

\subsection*{Bayesian Network}			 
			
We made our Bayesian network to help us find the enemy's spawn position. The main variable we wanted to look at is our opponent's scout. We can use information such as the time we see the scout and the position on the map where we see the scout to help us find the enemy base faster. 

EnemySpawn: This is our main variable we are trying to find. Ultimately all of the evidence we are collecting we are using to find this value. This is the reason I put it as the top node in the network (TimeSeenFrom and TimeSeenTo could be considered heads but hey only effect one variable each). Most of the other variable have direct links from this. Once we know the value of this we are satisfied and ultimately do not need the network any more because we do not care about predicting the other values.

OurSpawn: This node is a child of EnemySpawn. We can use the values in this node to determine where the enemy does not spawn. We simply say that the enemy cannot spawn where we spawn at. Simple logic I think. At the beginning of a match we know where we spawn and can instantly put evidence on the OurSpawn node.
 
EnemyNotAt(NE,SE,SW,NW): These nodes are as a way to keep track of position we know our opponent did not spawn. Since a node can only be in one state at a time, we made four nodes that can influence the probabilities in enemy spawn. Ways we would generally get these values could be our own scouts arriving at a position and not finding our opponent.

WorkerScoutPositionFrom: This is one of the most important variables. When we observe a worker coming from a certain position we can influence our belief on our opponents spawn position. TimingSeenFrom is a parent of this node. We use the information from this node to help our prediction. Just seeing an enemy at a certain position isn't enough, we need the time we see it to help us form our beliefs. When we obtain evidence on WorkerScoutPositionFrom we will also always gather information on TimingSeenFrom.

TimingSeenFrom: This node is used to help us use the information from WorkerScoutPositionFrom. It has the values: Almost None, Early, Middle, and Late. These values are the times we may see the opponent's scout. If we see it earlier then our beliefs will be different than if we see it later.

WorkerScoutPositionTo: Similar to WorkerScoutPositionFrom, but we are checking if we see the enemy scout going to a potential base.

TimingSeenTo: Similar to TimingSeenFrom, but we are checking the time we see the opponent scout go to a potential base.

OverlordDirection: This node is fairly simple and contributes a lot in predicting the spawn location. Since overlords are so slow, a player will be able to use the direction the overlord is coming from to predict the spawn location. By the time an overlord would have visited two bases we probably already know where the enemy's base is. This variable seems really useful, but it only helps when fighting against zerg players. The overlord can take a while to get to even its first base and by then we might have already needed to know where our opponent is. EnemySpawn is a parent of OverlordDirection. Once we know the direction the overlord is coming from we gain a lot of information on the enemy's spawn position.


