In this project we look into the \abrts game Starcraft Broodwar form the Machine Intelligence perspective. Starcraft Broodwar is interesting in this field for several reasons: it requires as much strategy as chess, it has the unknown aspect of poker, and it is as unpredictable as scrabble. These three factors makes it an ideal game for Machine Intelligence.

Another advantage of Starcraft Broodwar is that it is easy to learn but difficult to master. The basics of the game are very simple and can be understood in a matter 30 minutes, but it can take years of intensive training to become a true professional. \\\\

Using a computer to play Starcraft Broodwar is interesting since a computer exceeds humans in several areas. It will be able to do many calculations during a game, and be able to control the mouse and keyboard very rapidly. Specially since these mouse and keyboard movements open the door for some interesting opportunities. A human can only control a limited amount of units or buildings at a time, but a computer is completely able to control them simultaneously. \\

Machine Intelligence is interesting in relation to Starcraft Broodwar because we would be able to make our bot really good at just specific fragments of the game. These could be:
\begin{itemize}
	\item Teach it to use just the right strategy at the right time.
	\item Predict what the other players are during.
	\item Teach it to control single units in just the right way.
\end{itemize}

All the above parts make Starcraft Broodwar an interesting game to consider. We have a lot of different opportunity areas to focus on for the remainder of this report.


\section{Purpose}

	The purpose of this project is to create a computer program that is able to play a full game of Starcraft Broodwar. The computer should be able to perform tasks that would be impossible for a human being. It should be able to learn through trials from its mistakes and improve by using machine intelligence theory. 


\section{Problem Statement}
	Our goals for this project are to:
	\begin{itemize}
		\item Create an intelligent bot for the game Starcraft Broodwar
		\item Apply Machine Intelligence theory in the modelling of the bot
		\item Make a bot that can improve by playing Starcraft Broodwar
	\end{itemize}

\section{Overview}
	We will now give a brief overview of the different chapters of the report.

\paragraph*{Chapter 2 - Analysis}
The analysis begins with an introduction to the basic rules for playing a game of Starcraft Broodwar. Then we go on to explain some of the most common Terran tactics. We also talk about the importance of information and the balance between macro and micro. Then we take a look at what Machine Intelligence can be used to make a bot and look at which aspects of the game that can benefit the most from Machine Intelligence. The final thing in the analysis will be an analysis of the different units available for Terran and their strengths.

\paragraph*{Chapter 3 - Design}
In the design chapter we begin by describing the different criteria we have for our bot. Then we go on to explain how we designed the different managers we use for various tasks like building and scouting. Then we give an overview of what potential fields are and how we use them for movement. Afterwards we describe different kinds of agent learning and how we can use it to improve potential fields. At the end we talk about Bayesian networks and decision trees to make predictions and decisions.

\paragraph*{Chapter 4 - Implementation}
In this chapter we begin by looking at how we implemented Bayesian networks. Then we talk about the implementation of the managers and explain the most important parts of the code. We then look at potential fields and the differences between the design and implementation. At the end we look at how we implemented Q-Learning.

\paragraph*{Chapter 5 - Tests}
In chapter 5 we start by using a few basic test to compare against the potential fields. First we look at how the built in bot does against itself. Then we look at how our potential fields do without any learning against the built in bot. Then we look at how different alpha and gamma values effect the way our bot learns and how this effects its performance. After this we test the different Bayesian networks and how they do at predicting the enemy's spawn, predicting the enemy's build order, and the enemies threat level.

\paragraph*{Chapter 6 - Conclusion}
In this chapter we conclude on this project.

\paragraph*{Chapter 7 - Future Work}
This chapter focus on different actions we could take to improve our work if we had more time.
