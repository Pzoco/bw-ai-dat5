People make decisions every day that may seem trivial, but for computers these decisions would be very difficult. The reasoning we do 
is based on our preferences, needs, and experience. We learn from our past knowlege which we can later use to act on future events. One can argue that computers can
make decisions based on experience and learn from them as well.
For computers to make decisions like humans a lot of reflection is needed because computers can't think like people do, instead they can only perform a series of tasks.
One field where we could specify such task would be in video games, a field where \abai is commonly used. When playing 
a game you want to have your AI opponent to be as competitive as a human player to get the right challenge. \\

For this project we will be writing a bot for a game called Starcraft Broodwar. The game is a \abrts game, which requires a player to build an army 
which can defeat the opponent. One advantage a computer would have in Starcraft is the excessively high speed and precession required for a human player to master the game. 
This precession is difficult for humans to master but is simple to program a computer to do.
In Starcraft you can choose between three races: The insectoid Zerg, the advanced alien race Protoss and the Human-like Terran. 
The three races are all unique and excel at different skills, which requires players to adapt their strategies depending on which race they are facing.


\section{Purpose}
	The purpose of this project is to create a computer-bot which can play Starcraft Broodwar. 
	The bot is going to learn how to control a unit type, so that it will kill the opponent's army it else wouldn't have killed.
	The bot should be enable to react to what the opponent does by applying scouting.

\section{Problem Statement}
	Our goals for this project is to:
	\begin{itemize}
		\item Create an intelligent bot for the game Starcraft Broodwar
		\item Apply machine intelligence theory in the modeling of the bot
		\item Make a bot that can learn by playing Starcraft
		\item Make sure the bot analysis the battles and know if it should pull back.
		\item The bot should know a sample of built orders, so it can use the best strategy
		\item Make a bot that can compare the known variables from scouting and chose the best strategy
	\end{itemize}

\section{Overview}
	A short overview of the report where one can read what the different chapters contain

\subsection{Chapter 1 - }
Introduction to the project explaining how human makes decision and how a computer would make decisions. This part also explains the purpose of this project, and what kind of problem we would like to solve.
\subsection{Chapter 2}
Here we explain the basic rules of the game, what is important to obtain to win a match and how one can win over an opponent. We mention some different areas of the game one has to master to be good or even perfect to win, we also mention some different standard attack methods one can use to slow down the opponent. At last we analysis what kind of unit we want a computer to control.
\subsection{Chapter 3}
The design part is about different design methods one can use to control several units on a map, to avoid obstacles and still be drawn to the opponent. The last part is about the theory of agent learning what different kinds we have looked upon.
\subsection{Chapter 4}
This chapter is where the theory is being used in practice, and here we explain how we have implemented the potential fields, and how we have implemented the learning part of the agent.
\subsection{Chapter 5}
This chapter have the purpose to test our bot. We do this by having three test where each test is run on two different maps. The first test is the case
where the enemy attack the vultures without the vultures being controlled. This gives an idea of how strong the two forces are comparet with each other. The next test is the improvements our bot will have by using potential fields. The last test is about our bots ability to learn by using reinforcement learning.
\subsection{Chapter 6}
This chapter is where we will conclude the impact the machine intelligence have had on our bot and the results we have gathered from our bot.
\subsection{Chapter 7}