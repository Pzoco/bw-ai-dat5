\section{Aspects of winning}
	\subsection{Build order}
		A build order as an early game strategy describing in which order certain building and units should be produced\cite{wiki_build_order}. 
		The purpose of a build order is to optimizing the players economy and control the timing of units to give 
		the player the highest advantages at certain points in the game.\\
		The build order depends on many factors like the specific map played and the race of the different players in the map. 
		Early game scouting can help determain the other players build order, and coutering there tactic.\\
		
		Build order is mostly important at the beginning of the game, and depends a lot on timing. 
		The player must choose a spot to build a certain building but also ensure a worker is there to do it, 
		and make sure the player has the required recources. 
		A good player can time this. 
		So his worker reaches the construction site at the exact time he have enough minerals to actually build the building.\\
		
		Later in the the game the player must adapted strategy depending on the information recived about the other player.
	
	\subsection{Information}

		Every strategy and decision making throughout the game depends on the pieces of information 
		known in the current moment. From the most basic decisions like where to build, what to build and
		with which units; to wider decisions like how defend, start attacking, when to rush or how to counterattack.
		Everything depends on the information, the decision making process begins and grows with the input information.
		Therefore, how the non-player-character obtains this pieces of information is just as important 
		as how it process them. 
	
			\subsubsection{Knowing the map}
				During the game, many of the actions have to take into consideration the map. Movements, starting point, 
				tactics and general strategy depend on how much data there is about the map.  
				
				For example; the resources you need for building and expanding through the game are normally distributed
				equivalently between the players and the players are normally placed at opposite ends of the maps. 
				Considering this information and the information obtained in the moment the game starts 
				(map, size, exact location, opposite race) the initial build order and start actions can change. 
		
			\subsubsection{Scouting}
				Exploring the map is not only to know the geographical structure and find the resources needed, 
				but to obtain information on the actions of the opposite player. The player's actions will depend on
				the enemies' position, buildings, units, tactics and general strategy. The processing of the information 
				obtained form scouting is more important and delicate, since its obtaining smaler peces of information 
				that structure a larger network of unknown information. Also, this other source of information 
				is more volatile and temporal than the previous considerations; because it is only obtained for the 
				limited amount of time the unit stays in a certain place and to the areas the buildings limit the fog of war. 
	\subsection{Macro}
		Macro is one of the most imported aspect of starcraft brood war. For a player to win a game the player needs good macro. 
		Macro is the ability to use the resources and making sure all the buildings are working. 
		The most difficult part of marco is to keep up production of units and keeping down resources while attacting.  
			
			\subsubsection{Queing}
				If a player have good macro, he should not queue up units. 
				Because if a player is queing up units, the player can't use the resources that is binded the units in the queing. 
				This can lead to the player having less military.
	
			\subsubsection{Economy}
				Marco is used to inforce economy by keeping worker production. Normaly the player wants to keep building workers, to keep a good economy.
				But it is fine to stop building workers, if it is a part of the build.
				
				
	\subsection{Micro}
		Micro management is an important part of starcraft broodwar. 
		Micro-management requires a player to control their units so that you can get the most out of them. 
		There are several ways to do this.\\
		For a human player this can be very difficult as it requires a lot of concentration and can hurt an inexperienced player macro-management. 
		An advantage for a computer player is it can concentrate on macro and micro management and will not slip up.\\
		
		One way to do this is to keep units alive as long as possible. 
		The idea behind this is simple if your units stay alive after a battle then you do not have to buy as much units as your opponent. 
		Players sometimes move hurt units out of battle and then back in. 
		The reason for this is that the enemy will then start attacking another one of your 
		units and then your hurt unit can enter back in the battle and continue doing damage. 
		The idea behind this is that a unit will do the same amount of damage no matter how much health it has but will do no damage if it is dead.\\
		
		Another form of micro is making several of your units attack one unit at a time. 
		A computer can do very well at this because it can perform calculations so that is knows how many units it takes to kill another unit in one shot.
		This is usefult so that too many units don't waste their shot in killing one unit when they can be doing damage to other units.\\
		In order to micro units effectively you must be do a lot of different actions at one time. 
		This is generally measured by a unit called APM (actions per minute). 
		A computer can have an extremely high APM. Making it easy for it to micro its units perfectly.
		
	\subsection{Unpredictability}
		When playing tournaments or just playing a normal match being unpredictable can win you the game. 
		If your opponent is adapting to your play style you can put him off by doing something different or playing strange.
		In tournaments psychology is a big factor, professional players can make the other player do a certain strategy because they know 
		how the other player will react to certain things. This feature will be hard to near impossible to learn the bot, because the bot won't be able to 
		identify the opponent's playstyle. If it can not identify the playstyle it will not be able to put the opponent off. Our goal is for the bot 
		to win against other bots, so the aspect of unpredictability may not be important at all. 
