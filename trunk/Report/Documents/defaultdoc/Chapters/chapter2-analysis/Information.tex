\section{Information}

Every strategy and decision making throughout the game depends on the pieces of information 
known in the current moment. From the most basic decisions like where to build, what to build and
with which units; to wider decisions like how defend, start attacking, when to rush or how to counterattack.
Everything depends on the information, the decision making process begins and grows with the input information.
Therefore, how the non-player-character obtains this pieces of information is just as important 
as how it process them. 

\subsection{Knowing the map}
During the game, many of the actions have to take into consideration the map. Movements, starting point, 
tactics and general strategy depend on how much data there is about the map.  

For example; the resources you need for building and expanding through the game are normally distributed
equivalently between the players and the players are normally placed at opposite ends of the maps. 
Considering this information and the information obtained in the moment the game starts 
(map, size, exact location, opposite race) the initial build order and start actions can change. 


\subsection{Scouting}

Exploring the map is not only to know the geographical structure and find the resources needed, 
but to obtain information on the actions of the opposite player. The player's actions will depend on
the enemies' position, buildings, units, tactics and general strategy. The processing of the information 
obtained form scouting is more important and delicate, since its obtaining smaler peces of information 
that structure a larger network of unknown information. Also, this other source of information 
is more volatile and temporal than the previous considerations; because it is only obtained for the 
limited amount of time the unit stays in a certain place and to the areas the buildings limit the fog of war. 

