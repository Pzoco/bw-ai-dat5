\section{Bot analysis}
	This section will discuss in which way we will apply machine intelligence to control our bot.
	
	The most interesting thing about a bot that can play a game is that if it is able to outplay a 
	human player in either outplaying psychically or outsmarting mentally. 
	
	\subsection*{Outplaying a human player}
		A bot can psychically outplay a human player easily in Starcraft. 
		Doing micromanagement is really demanding for a human, because it requires a lot of actions and fast reaction. A bot may be able to control each 
		attacking unit individually and making intelligent decisions on attacking according to health, cooldown or other factors. This is for the most 
		impossible to do in a large scale for a human and that is why a bot that is focusing on micromanagement could be interesting to make. Another point 
		is that while a player is doing micromanagement a lot of other actions are needed elsewhere. If these tasks are handled this could lose one the game.
		
	\subsection*{Outsmarting a human player}
		Outsmarting refers to either bluffing, tricking or just making good decisions. This will be hard for a bot to do, because there are so many 
		different strategies and being able to reacting to all and even be able to counter these strategies will be impossible. Something unexpected 
		can break the bot down completely.
		
	\subsection*{Conclusion}
		The bot will be focusing on micromanagement, because it is interesting in the sense that it will be able to do something that is impossible for a 
		human player. It will also be easier to spot the learning part of the bot, because it is something that can be measured, e.g. units killed.
	