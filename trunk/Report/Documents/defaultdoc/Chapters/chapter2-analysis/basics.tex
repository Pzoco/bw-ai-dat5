%Before we can make a bot, we have to define how the game works and how a player can win.
In this chapter we will define the basic rules of the game and the techniques a player can use to help him achieve victory.
This will help us determine in which way we can apply machine intelligence theory, such that the bot can win a game of Starcraft. 

\section{Basic Game Rules}
	Starcraft is a complex game in which a player battles his army against his opponent's army. 
	There are many strategies involved in doing this, but the game can be summarized in some simple terms.\\
	Before the match starts the player chooses one of three different races: Protoss, Terran, or Zerg. 
	Even though all of these races are unique with different strategies and units, 
	the basic strategies and techniques of the game remain the same no matter which one you choose.\\	
	The first rule of the game is that if a player loses all of his buildings then he loses. 
	This means that in order to win against an opponent you must kill his last building before he kills your last building. \\	
	At the beginning of the game you start out with one building that can only train the most basic worker unit which can collect resources. 
	Resources are used to buy buildings, units, and upgrades. 
	It is important to collect  as much resources as possible so you can buy more buildings, units, and upgrades than your opponent.
	This can be a difficult task. It is not enough just to get more resources than your opponent. 
	You need to utilize your resources, resources that are not being used for anything are wasted. 
	In general you want to have a high income but keep a low amount of resources in reserve. 
	If you have extra resources you need to either buy more buildings, train more units, or buy upgrades for your army.\\
	In order to destroy your opponent's building you must have an army (or at least some units). 
	In the beginning of the game you have almost no army and you want to make your army stronger as the game progresses. 
	It is not good enough to just have an army, but you must have a better army than your opponent, when an confrontation occur.
	The optimal balance in army size is the exact force that is needed to keep you alive and force the enemy back, if an attack occurs.\\	
	Generally there is a trade off between your army and your economy. 
	The larger the army you have the less you are spending on growing your economy (and vice versa). 
	If you spend too much on your army but not enough on economy you may become too far behind later in the game to win. 
	On the other hand if you spend all of your resources on your economy the enemy army could come and wipe you out. A picture from the beginning of a normal game can be seen in figure \ref{fig:normalgame} in section \ref{label:normalgame} in the appendix.\\

 