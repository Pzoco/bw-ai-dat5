\section{Bayesian Networks and Decision Trees}
In this section we are going to define Bayesian networks and decision threes. Then we are going to compare the two decision models and choose
the best model our bot can use for scouting analysis.

\subsection{Bayesian Networks}
	Bayesian networks are simple graphical models, where each probability for the variable is calculated. Therefore the Bayesian networks is used for
	calculating new probabilities, whenever new information is gathered. So a Bayesian networks have a  set of variable and
	these variable is connected with directed edges. Each variable in the Bayesian network must have a finite set of mutually exclusive states. By this we	
	mean that :::. To make sure that we can calculator a result, the Bayesian network needs to be a acyclic directed graph. A variable needs a conditional 		
	probability	table for each of it's

\subsection{Decision Trees} parents. This means that the amount of calculations in a Bayesian networks depends on the number of variables and the 		edges that connects them.
	Decision trees are used to represent decision problems. A decision tree consists of three types of nodes: decision nodes, chance nodes and utility nodes. 
	The link from a decision node to a chance node is called an action, and a link from a chance node to a decision node is called a state. 
	The idea of the decision tree is to find the path that will give us the highest utility (reward), so to make a decision tree over a decision problem, 
	every possible path of decisions have to be shown in the tree. This will give a tree that grows exponentially with the number of decision and 
	chance variables, so given small decision problems will require big trees, if not reduced. 
	
	A decision tree could be made for predicting the enemy's spawn location or their strategy. 
	
\subsection{Conclusion}
