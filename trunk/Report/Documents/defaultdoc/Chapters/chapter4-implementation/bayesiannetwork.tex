This chapter will explain implementation of the core parts of the bot. This is done by showing code snippets and examples.

For our code to talk to Starcraft Broodwar we use an API called BWAPI, a link can be found in the bibliography \cite{bwapi}.


\section{Class BayesianNetwork}
	For handling Bayesian networks we use the Hugin API, which can handle loading, inserting evidence and other manipulation of .net files. To actually 
	insert evidence a lot of steps is needed, so to make it easier for classes to use the networks we wrote a class called BayesianNetwork.

	When a class needs a Bayesian network it uses this class. The constructor takes a filename, which then loads the given 
	Bayesian network. The class also contains methods for printing nodes, retracting and inserting of evidence and getting the probability of a state. 
	The method for inserting evidence will be explained beneath.
	\subsection*{Inserting evidence}
		\begin{Sourcecode}[caption=EnterEvidence method]
void BayesianNetwork::EnterEvidence(std::string nodeName,std::string stateName)
{
domain->uncompile();
NodeList nodes = domain->getNodes();
for (NodeList::const_iterator it = nodes.begin(); it != nodes.end(); ++it)
{
	Node* node = *it;
	if(nodeName == node->getName())
	{
		(*@\lnote@*)DiscreteChanceNode* evidenceNode = dynamic_cast<DiscreteChanceNode*>(node);
		size_t index = evidenceNode->getStateIndex(stateName);
		(*@\lnote@*)evidenceNode->selectState(index);
		break;
	}
}
domain->compile();
domain->propagate(H_EQUILIBRIUM_SUM, H_MODE_NORMAL);
}
		\end{Sourcecode}
		The method starts by uncompling which is needed to manipulate nodes. After this it loops through all the nodes until it finds the specified 
		node. At \lnnum{1} the node is converted to a DiscreteChanceNode, because evidence cannot be presented to a normal node. The reason for this 
		is that there are many types which inherit from the class Node, which does not all use evidence. The index of the wanted state is being 
		retrieved and the state is selected at \lnnum{2}, which is the same at presenting evidence at the state. After the evidence is presented, the 
		domain is then compiled and propagated. The propagate function calculates the new probabilities for the states in the network.