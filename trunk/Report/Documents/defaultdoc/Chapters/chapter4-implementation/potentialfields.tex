
\section{Implementing Potential Fields}
	In this section we are going to take a look at the implementation of the potential field. First we will look at the main loop and the general structure. Then we move on to each part of the potential field itself.
	
	\subsection{General Structure}
	The potential field is calculated for individual units. These units belong to a squad. When we calculate the potential field we also pass a reference to the squad called \textit{mySquad} because it is needed to calculate \textit{Squad center} and \textit{Ally potential}.
\pagebreak
\begin{Sourcecode}[caption=Main loop]	
(*@\lnote@*)BaseTactic::InitializeParameters(mySquad);
Position centerPosition = _unit->getPosition();
(*@\lnote@*)std::list<BWAPI::Position> listOfPositions = MathHelper::GetSurroundingPositions(centerPosition,48);
BWAPI::Position bestQPosistion = BWAPI::Position(1,1);
(*@\lnote@*)double centerQ = BaseTactic::CalculateQPotentialField(_unit->getPosition(),false);
double higestQfound; 
bool firstCalculation = true;
(*@\lnote@*)for each(BWAPI::Position position in listOfPositions)
{
	double currentQ = BaseTactic::CalculateQPotentialField(position,false);
	(*@\lnote@*)if(firstCalculation)
	{
		firstCalculation = false;
		higestQfound = currentQ;
		bestQPosistion = position;
	} 
	if(currentQ > higestQfound)
	{
		higestQfound = currentQ;
		bestQPosistion = position;
	}
	//used to print the potential around the unit
	Broodwar->drawTextMap(position.x(),position.y(),"%d",(int)currentQ); 
}
(*@\lnote@*)if(centerQ == higestQfound)
{
	bestQPosistion = centerPosition;
}
\end{Sourcecode}	
	In \lnnum{1} we initialize all the parameters because they are the same for all the tiles:
	\begin{itemize}
	    \item int da - Is the distance to closed ally unit.
		\item int ds - The distance from center of army to unit.
		\item int sv - The units maximum shooting range. -1 if there is no weapon for this type.
		\item int sva - The units maximum shooting range for air. -1 if there is no weapon for this type.
		\item int de - Distance to nearest known enemy.
		\item bool wr - A boolean denoting whether or not the weapons are ready to fire.
		\item BWAPI::Position squadPos - The center of the squad.
	\end{itemize}
	They are all used in the calculation of the potential field later in this section.	
	
    Next we need to find the tiles we want to calculate the potential for. This is done using the \textit{MathHelper::GetSurroundingPositions}  function in \lnnum{2}. This function takes two parameters, the tile the unit is currently on and the distance to the center of the adjacent tiles. From this we calculate all tiles around the unit and and them to a list. \\

    Now in \lnnum{3} we calculate the potential for the center tile. This is used in \lnnum{6} to compare it to the best potential found. If they are equal we choose the center tile. This is a rare case and mostly only occurs when all the potential fields are zero. In that case we do not want to move. If we did not choose to make \textit{bestQPosistion} the \textit{centerPosition} and returned any other tile, the unit would move to this tile. \\
    
    In \lnnum{4} we have the main loop. This loop iterates trough all the tiles around our unit and selects the highest one. There is one special case in \lnnum{5} which is the first calculation. In this calculation we always select it as the current best option. This is done so we have something to compare with rather than just set it to an arbitrary tile and number. \\
	
	    
	In the calculation of the potential field we use three different constants listed below. These constants are initialized to a parameter we found by trial and error.
	\begin{itemize}
		\item \_variables.SQUADDISTANCE\_CONSTANT - Is set to $150$ px.
		\item \_variables.ALLYDISTANCE\_CONSTANT - Is set to $50$ px.
		\item \_variables.EDGESDISTANCE\_CONSTANT - Is set to $250$ px.
	\end{itemize}
	
	An important thing to note before we look at the individual function is a small change from the design. In the design we use negative forces in repulsive fields, but due to the reinforcement learning this has changed. The reinforcement learning will change the forces, and it will be able to make them both positive and negative. In other words the reinforcement learning will be able to change whether or not a potential field is attractive or repulsive. 
		
	\subsection{Squad Center}	
		\begin{Sourcecode}[caption=Squad center]
(*@\lnote@*)_parameters.dsv = pos.getApproxDistance(_parameters.squadPos);	
int useSquad = 0;
(*@\lnote@*)if(_parameters.ds > _variables.SQUADDISTANCE_CONSTANT)
	useSquad = 1;

(*@\lnote@*)squad += ((double)_variables.FORCESQUAD * (2*(double)_parameters.ds-(double)_parameters.dsv))*useSquad;
(*@\lnote@*)squadQ += (2*(double)_parameters.ds-(double)_parameters.dsv)*useSquad;
\end{Sourcecode}
		In \lnnum{1} we calculate \textit{dsv} by using the build in function getApproxDistance which returns the distance between two tiles as an integer. \textit{pos} is the current tile we want to calculate the potential for and \textit{squadPos} is the tile in the center of the squad. \\
		
		Then in \lnnum{2} we check weather or not to use the \textit{Squad center} which depends on the distance from the unit, know as \textit{ds}, to the center of the squad. This is used to handle the two cases found in the design of the potential field. \\
		
		Afterwards we calculate the potential in \lnnum{3} where \textit{\_variables.FORCESQUAD} is $f_{S}$. \textit{\_variables.FORCESQUAD} is learned by the reinforcement learning in section \ref{section:reinforcement}. 
		Unlike in the design we multiply useSquad on the end. This is to make it zero if we do not want to use potential field which would make the entire calculation become zero. This is way to control that the use of the potential field is not only used in \textit{Squad center} but also in all the other potential fields. \\
		
		\textit{squadQ} in \lnnum{4} and likewise will be explained in the implementation of reinforcement learning in section \ref{section:reinforcement}.
		
		The \textit{Squad center} implemented and running can be seen in figure \ref{fig:SCA}.
\insertmarginfigure{height=2in}{PotentialfieldsImplementation/Squadcenter.png}
			{Squad center}{fig:SCA}{-3in}
			
	\subsection{Maximum Distance Positioning}
		\begin{Sourcecode}[caption=Maximum distance]
(*@\lnote@*)int distanceToEnemyFromUnit = _parameters.de;
(*@\lnote@*)_parameters.due = MathHelper::GetDistanceToNearestEnemy(pos);
(*@\lnote@*)int correctedDistance = (2*_parameters.de - _parameters.due);
int useMaxDist = 1;
if(_parameters.sv > distanceToEnemyFromUnit)
	useMaxDist = 0;

(*@\lnote@*)maxdist += ((double)_variables.FORCEMAXDIST * (double)correctedDistance)*(double)useMaxDist;
maxdistQ += (double)(correctedDistance)*(double)useMaxDist;
\end{Sourcecode}
	%TEXT
	
	In \textit{Maximum distance positioning} we start by calculating \textit{de} and \textit{due}in \lnnum{1} \lnnum{2}. These are used to calculate $(2*de - due)$ in \lnnum{3}, known in the code as \textit{correctedDistance}. \\
	
	\textit{correctedDistance} is being calculated even if we do not use \textit{Maximum distance positioning} because it is also used in \textit{Weapon cool down}. \\
	
	In \lnnum{1} and \lnnum{2} we make use of \textit{MathHelper::GetDistanceToNearestEnemy} which is a function that takes a tile as input and then iterates trough all visible enemies. Then it calculates the distance to the nearest enemy and returns it.
	
	The last thing to do is calculate the potential which is done in \lnnum{4}. The \textit{Maximum distance positioning} implemented and running can be seen in figure \ref{fig:MDP}. \\

	
	\insertmarginfigure{height=2in}{PotentialfieldsImplementation/MaximumDistance.png}
			{Maximum distance positioning}{fig:MDP}{-3in}

	\subsection{Weapon Cool Down}
		\begin{Sourcecode}[caption=Weapon cool down]
int toCool = 1;
if(_parameters.wr)
	toCool = 0;
	
cool += _variables.FORCECOOLDOWN*correctedDistance*toCool;
coolQ += correctedDistance*toCool;
\end{Sourcecode}	
	\textit{Weapon cool down} looks a lot like \textit{Maximum distance positioning} and does almost the same thing. The only difference is when they are applied. An example of \textit{Weapon cool down} can be found in figure \ref{fig:WCD}.
	
	\insertmarginfigure{height=1.5in}{PotentialfieldsImplementation/Weaponscooldown.png}
		{Repulsion from Weapons cool down}{fig:WCD}{-3in}

	\subsection{Ally Units}
		\begin{Sourcecode}[caption=Ally units]
(*@\lnote@*)_parameters.dua = MathHelper::GetDistanceToNearestAlly(pos,_unit->getID());
int useAlly = 0;
if(_parameters.da < _variables.ALLYDISTANCE_CONSTANT)
	useAlly = 1;

ally += (double)_variables.FORCEALLY*(double)(2*_parameters.da - _parameters.dua)*useAlly;
allyQ += (double)(2*_parameters.da - _parameters.dua)*useAlly;
\end{Sourcecode}	
	In \textit{Ally units} at \lnnum{1} we use a function called \textit{MathHelper::GetDistanceToNearestAlly} which takes as input a tile position and the id of the current unit. We need the unit id because the function iterates trough all our units, and we want to avoid calculating the distance to the current unit. If we did not use the unit id we would always get the distance from the current unit to itself because the distance would always be less than the distance to other units. \textit{Ally units} can be seen in figure \ref{fig:AU}.
	
		\insertmarginfigure{height=1.5in}{PotentialfieldsImplementation/Allyunits.png}
			{Repulsion of ally units}{fig:AU}{-3in}

		\pagebreak
	\subsection{Edges and Cliffs (Repulsive)}
	\begin{Sourcecode}[caption=Edges and cliffs]
(*@\lnote@*)_parameters.duc= (int)MathHelper::GetDistanceBetweenPositions(BWTA::getNearestUnwalkablePosition(pos),pos);
(*@\lnote@*)_parameters.dc = (int)MathHelper::GetDistanceBetweenPositions(BWTA::getNearestUnwalkablePosition(pos),_unit->getPosition());

int useEdge = 1;
if(_parameters.duc > _variables.EDGESDISTANCE_CONSTANT)
	useEdge = 0;

edge += (_variables.FORCEEDGE)*(2*_parameters.dc-_parameters.duc)*useEdge;
edgeQ += (2*_parameters.dc-_parameters.duc)*useEdge;
\end{Sourcecode}	
    In \lnnum{1} we use the function \textit{MathHelper::GetDistanceBetweenPositions} to get the distance between the nearest unwalkable position and the current tile. To get the nearest unwalkable position we use the built in function \textit{BWTA::getNearestUnwalkablePosition} which takes as input a position and then find the nearest unwalkable position. \\

	
	Normally \textit{dc} from \lnnum{2} would be calculated in the initialization, but because \\ \textit{BWTA::getNearestUnwalkablePosition} should depend on the current tile we have to recalculate it for each tile. The reason is that for each tile the nearest unwalkable position can differ, and we need to use the same unwalkable position in the calculation of the center tile. If we didn't do this, there could be cases where 
	we took the wrong tiles into consideration. An example could be a corner, where some tile would depend on one side, and other tiles on the other side. 
	The \textit{Edges and cliffs} can be seen in figure \ref{fig:EC}

	\insertmarginfigure{height=2in}{PotentialfieldsImplementation/EdgesAndCliffs.png}
			{Repulsion from cliffs}{fig:EC}{-3in}
