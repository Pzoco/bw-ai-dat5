\section{Implementing potential fields}
	
	\subsection{General structure}
	%TALK ABOUT 
	initialization.
	 _variables
	 _parameters
	 
	
	\subsection{Squad center}	
		\begin{Sourcecode}[caption=Squad center]
(*@\lnote@*)int dsv = pos.getApproxDistance(_parameters.squadPos);	
int useSquad = 0;
(*@\lnote@*)if(_parameters.ds > _variables.SQUADDISTANCE_CONSTANT)
	useSquad = 1;

(*@\lnote@*)squad += (double)(_variables.FORCESQUAD * (2*(double)_parameters.ds-(double)dsv))*useSquad;
(*@\lnote@*)squadQ += (2*(double)_parameters.ds-(double)dsv)*useSquad;
\end{Sourcecode}
		In \lnnum{1} we calculate \textit{dsv} by useing the build in function getApproxDistance which returns the distance between two tiles as an integer.
		Then in \lnnum{2} we check weather or not to use the \textit{Squad center} which depends on the the distance to the center of the squad.
		After we done this we calculate the potential in \lnnum{3}. The only difference between .

\insertmarginfigure{height=2in}{PotentialfieldsImplementation/Squadcenter.png}
			{Squad center}{fig:SCA}{-3in}
			
	\subsection{Maximum distance positioning}
		\begin{Sourcecode}[caption=Maximum distance]
Position unitPos = _unit->getPosition();
_qParameters.de = MathHelper::GetDistanceToNearestEnemy(unitPos);
int distanceToEnemyFromUnit = _qParameters.de;
int due = MathHelper::GetDistanceToNearestEnemy(pos);
int correctedDistance = (2*_qParameters.de - due);
int useMaxDist = 1;
if(_parameters.sv > distanceToEnemyFromUnit)
	useMaxDist = 0;


maxdist += ((double)_variables.FORCEMAXDIST * (double)correctedDistance)*(double)useMaxDist;
maxdistQ += (double)(correctedDistance)*(double)useMaxDist;
\end{Sourcecode}
	%TEXT
	
	\insertmarginfigure{height=2in}{PotentialfieldsImplementation/MaximumDistance.png}
			{Maximum distance positioning}{fig:MDP}{-3in}

	\subsection{Weapon cool down}
		\begin{Sourcecode}[caption=Weapon cool down]
int toCool = 1;
if(_parameters.wr)
	toCool = 0;
	
cool += _variables.FORCECOOLDOWN*correctedDistance*toCool;
coolQ += correctedDistance*toCool;
\end{Sourcecode}	
	%TEXT
	\insertmarginfigure{height=2in}{PotentialfieldsImplementation/Weaponscooldown.png}
		{Repulsion of ally units}{fig:AU}{-3in}

	\subsection{Ally units}
	
	
		\begin{Sourcecode}[caption=Ally units]
int dua = MathHelper::GetDistanceToNearestAlly(pos,_unit->getID());
_parameters.da = MathHelper::GetDistanceToNearestAlly(pos,_unit->getID());

int useAlly = 0;
if(_parameters.da < _variables.ALLYDISTANCE_CONSTANT)
	useAlly = 1;
	

ally += (double)_variables.FORCEALLY*(double)(2*_parameters.da - dua)*useAlly;
allyQ += (double)(2*_parameters.da - dua)*useAlly;
\end{Sourcecode}	
	%TEXT
		\insertmarginfigure{height=2in}{PotentialfieldsImplementation/Allyunits.png}
			{Repulsion of ally units}{fig:AU}{-3in}

	\subsection{Edges and cliffs (Repulsive)}
	\begin{Sourcecode}[caption=Edges and cliffs]
int duc= (int)MathHelper::GetDistanceBetweenPositions(BWTA::getNearestUnwalkablePosition(pos),pos);
_qParameters.dc = (int)MathHelper::GetDistanceBetweenPositions(BWTA::getNearestUnwalkablePosition(pos),_unit->getPosition());

int useEdge = 1;
if(duc > _variables.EDGESDISTANCE_CONSTANT)
	useEdge = 0;

edge += (_variables.FORCEEDGE)*(2*_qParameters.dc-duc)*useEdge;
edgeQ += (2*_qParameters.dc-duc)*useEdge;
\end{Sourcecode}	

	%TEXT


	\insertmarginfigure{height=2in}{PotentialfieldsImplementation/EdgesAndCliffs.png}
			{Repulsion from cliffs}{fig:AU}{-3in}